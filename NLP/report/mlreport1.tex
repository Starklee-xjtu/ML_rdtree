%%%%%%%%%%%%%%%%%%%%%%%%%%%%%%%%%%%%%%%%%%%%%%
% An example of a lab report write-up.
%%%%%%%%%%%%%%%%%%%%%%%%%%%%%%%%%%%%%%%%%%%%%%
% This is a combination of several labs that I have done in the past for
% Computer Engineering, so it is not to be taken literally, but instead used as
% a great starting template for your own lab write up.  When creating this
% template, I tried to keep in mind all of the functions and functionality of
% LaTeX that I spent a lot of time researching and using in my lab reports and
% include them here so that it is fairly easy for students first learning LaTeX
% to jump on in and get immediate results.  However, I do assume that the
% person using this guide has already created at least a "Hello World" PDF
% document using LaTeX (which means it's installed and ready to go).
%
% My preference for developing in LaTeX is to use the LaTeX Plugin for gedit in
% Linux.  There are others for Mac and Windows as well (particularly MikTeX).
% Another excellent plugin is the Calc2LaTeX plugin for the OpenOffice suite.
% It makes it very easy to create a large table very quickly.
%
% Professors have different tastes for how they want the lab write-ups done, so
% check with the section layout for your class and create a template file for
% each class (my recommendation).
%
% Also, there is a list of common commands at the bottom of this document.  Use
% these as a quick reference.  If you'd like more, you can view the "LaTeX Cheat
% Sheet.pdf" included with this template material.
%
% (c) 2009 Derek R. Hildreth <derek@derekhildreth.com> http://www.derekhildreth.com
% This work is licensed under the Creative Commons Attribution-NonCommercial-ShareAlike License. To view a copy of this license, visit http://creativecommons.org/licenses/by-nc-sa/1.0/ or send a letter to Creative Commons, 559 Nathan Abbott Way, Stanford, California 94305, USA.
%%%%%%%%%%%%%%%%%%%%%%%%%%%%%%%%%%%%%%%%%%%%%%

\input kvmacros % For Karnaugh Maps (K-Maps)
\documentclass[UTF8]{ctexart}
\usepackage{graphicx} % For images
\usepackage{float}    % For tables and other floats
\usepackage{verbatim} % For comments and other
\usepackage{amsmath}  % For math
\usepackage{amssymb}  % For more math
\usepackage{fullpage} % Set margins and place page numbers at bottom center
\usepackage{listings} % For source code
\usepackage{subfig}   % For subfigures
\usepackage[usenames,dvipsnames]{color} % For colors and names
\usepackage{hyperref}           % For hyperlinks and indexing the PDF
\hypersetup{ % play with the different link colors here
    colorlinks,
    citecolor=blue,
    filecolor=blue,
    linkcolor=blue,
    urlcolor=blue % set to black to prevent printing blue links
}

\definecolor{mygrey}{gray}{.96} % Light Grey
\lstset{
	language=[ISO]C++,              % choose the language of the code ("language=Verilog" is popular as well)
   tabsize=3,							  % sets the size of the tabs in spaces (1 Tab is replaced with 3 spaces)
	basicstyle=\tiny,               % the size of the fonts that are used for the code
	numbers=left,                   % where to put the line-numbers
	numberstyle=\tiny,              % the size of the fonts that are used for the line-numbers
	stepnumber=2,                   % the step between two line-numbers. If it's 1 each line will be numbered
	numbersep=5pt,                  % how far the line-numbers are from the code
	backgroundcolor=\color{mygrey}, % choose the background color. You must add \usepackage{color}
	%showspaces=false,              % show spaces adding particular underscores
	%showstringspaces=false,        % underline spaces within strings
	%showtabs=false,                % show tabs within strings adding particular underscores
	frame=single,	                 % adds a frame around the code
	tabsize=3,	                    % sets default tabsize to 2 spaces
	captionpos=b,                   % sets the caption-position to bottom
	breaklines=true,                % sets automatic line breaking
	breakatwhitespace=false,        % sets if automatic breaks should only happen at whitespace
	%escapeinside={\%*}{*)},        % if you want to add a comment within your code
	commentstyle=\color{BrickRed}   % sets the comment style
}

% Make units a little nicer looking and faster to type
\newcommand{\Hz}{\textsl{Hz}}
\newcommand{\KHz}{\textsl{KHz}}
\newcommand{\MHz}{\textsl{MHz}}
\newcommand{\GHz}{\textsl{GHz}}
\newcommand{\ns}{\textsl{ns}}
\newcommand{\ms}{\textsl{ms}}
\newcommand{\s}{\textsl{s}}



% TITLE PAGE CONTENT %%%%%%%%%%%%%%%%%%%%%%%%
% Remember to fill this section out for each
% lab write-up.
%%%%%%%%%%%%%%%%%%%%%%%%%%%%%%%%%%%%%%%%%%%%%

\newcommand{\labtitle}{NLP项目分类问题}
\newcommand{\authorname}{李梓铉}
\newcommand{\professor}{胡飞虎教授}
\newcommand{\classno}{3118103163}
% END TITLE PAGE CONTENT %%%%%%%%%%%%%%%%%%%%


\begin{document}  % START THE DOCUMENT!


% TITLE PAGE %%%%%%%%%%%%%%%%%%%%%%%%%%%%%%%%%%%%%%
% If you'd like to change the content of this,
% do it in the "TITLE PAGE CONTENT" directly above
% this message
%%%%%%%%%%%%%%%%%%%%%%%%%%%%%%%%%%%%%%%%%%%%%%%%%%%
\begin{titlepage}
\begin{center}
{\LARGE \textsc{大数据与深度学习应用} \\ \vspace{4pt}}
{\Large \textsc{\labtitle} \\ \vspace{4pt}}
\rule[13pt]{\textwidth}{1pt} \\ \vspace{150pt}
{\large  \authorname \\ \vspace{10pt}
学号 \classno\\ \vspace{10pt}
指导老师: \professor \\ \vspace{10pt}
\today}
\end{center}
\end{titlepage}
% END TITLE PAGE %%%%%%%%%%%%%%%%%%%%%%%%%%%%%%%%%%





%%%%%%%%%%%%%%%%%%%%%%%%%%%%%%
%%%%%%%%%%%%%%%%%%%%%%%%%%%%%%
\section{问题背景}
%No Text Here
%%%%%%%%%%%%%%%%%%%%%%%%%%%%%%%
\subsection{背景介绍}
\begin{comment}
This is a lab template which has a ton of different things which are useful in writing lab write-ups in the Computer Eningeering field.  This is demonstrating the comment block. Don't be overwhelmed, it may seem like a lot to take in at a time, but it's worth spending the time learning it.
\end{comment}
文本分类作为NLP的常见任务,具有很高的实际应用价值。本文将采用LSTM模型,训练一个能够识别文本不同类别项目描述的分类器。项目描述包含项目名称,项目简介,主要产品,业主方,和所在地。项目类别包含7类如建筑化工等。本文没有考虑不同项目描述之间的区别,直接将所有描述整合为同一类进行分类。意思就是将项目名称项目简介等对应的string连接为同一个string作为输入。
本文使用keras深度学习库实现了基本的LSTM做分类任务的流程,但由于时间有限的只是完成了一个标准的流程,并熟悉了调参的过程。在本文提到的模型之上,还想到有一些可以改进的地方,以后有空的话,笔者可以再做优化,比如针对不同描述各训练一个模型并进行投票得到一个集成模型。

(Keras是一个高层神经网络API,Keras由纯Python编写而成并基Tensorflow、Theano以及CNTK后端。Keras 为支持快速实验而生,能够把idea迅速转换为结果,)

如何使用训练好的模型请看 README.md
\vspace{3mm}

\clearpage

\section{解题思路}
\subsection{文本表示与特征提取}
使用word2vec算法,用高维向量(词向量,Word Embedding)表示词语,并把相近意思的词语放在相近的位置,而且用的是实数向量(不局限于整数)。我们只需要有大量的某语言的语料,就可以用它来训练模型,获得词向量。从而实现文本的表示与特征提取,并且词向量很方便做聚类,定义距离即可得到相近意思的词语,
通过训练得到词向量的转化,就可以将句子分词后得到多个词向量并使用LSTM处理。或者用传统方法如SVM直接根据词向量进行分类。
本文使用jieba分词与Gensim库实现分词和词向量的处理。
\subsection{分类算法}
分类算法部分使用了两种分类算法,第一种是支持向量机,对将句子转化为词向量后直接使用SVM求距离直接进行分类。第二类是LSTM神经网络。将矩阵形式的输入(原始句子)转化为多个较低维度一维向量(词向量),并利用LSTM的特性,使用遗忘门控制训练时候梯度的收敛性的同时,也能够保持长期的记忆性,实现语义的理解分类。
\begin{figure}[H]
  \centering
  \label{fig:Per6A}\includegraphics[width=0.8\textwidth]{lstm.png}\
  \caption{LSTM流程图}
  \label{fig:oscil}
\end{figure}

\clearpage %newpage会被忽略因为数字会导致直接

\section{结果与讨论}
\subsection{支持向量机SVM}
去掉包含空值nah的数据共有2405条可用数据,训练集包含1923个数据,测试集包含481个数据,个数比为4:1,共经过159次迭代后,SVM分类可以在达到0.7的分类准确度。
\begin{figure}[H]
  \centering
  \label{fig:Per6A}\includegraphics[width=0.4\textwidth]{svm.png}\
  \caption{SVM结果}
  \label{fig:oscil}
\end{figure}
这一步可以确定词向量的训练应该是没有问题的,并且对深度学习的效果提了一个benchmark,至少高于70\% 的准确率才可以说明这个训练的神经网络是足够高效的。

\subsection{长短时记忆网络LSTM}
数据预处理部分考虑不同类别的项目个数区别过大会影响模型的准确性,因此在预处理的时候,统计了每个类别的个数,并以个数最少的那类项目的个数为基准,在其他类中取同样个数的数据来构成训练集。
经统计[建筑', '化工医药', '能源矿产', '材料制造', '其他', '电力', '交通运输']类别对应个数为[344, 209, 280, 183, 446, 706, 236],因此取最少的类别个数183,每一类中取183个数据,得到共1281个数据,按4:1划分训练集和测试集。

lstm激活函数使用sigmoid,并使用EarlyStopping回调函数(效果是在val\_accuracy 10 个epoch后仍然不上升的情况下终止训练),以及dropout层减轻过拟合。


\begin{figure}[H]
  \centering
  \subfloat[1 layer LSTM]{\label{fig:Per4}\includegraphics[width=0.4\textwidth]{final_accuracy.png}} \\
  \subfloat[2 layer LSTM]{\label{fig:Per5}\includegraphics[width=0.4\textwidth]{2layer_accuracy.png}}
  \subfloat[3 layer LSTM]{\label{fig:Per6}\includegraphics[width=0.4\textwidth]{3layer_accuracy.png}}
  \caption{不同层数LSTM训练过程}
  \label{fig:oscil}
\end{figure}
调参主要从不同词向量个数,不同网络层数,不同dropout 参数。
通过调参得到如下训练结果。训练结果按照分类准确度从高到低排列如下。颜色表示模型的优劣,绿色最优,红色最差。

\begin{figure}[H]
  \centering
  \label{fig:Per6A}\includegraphics[width=1\textwidth]{excelRS.png}\
  \caption{LSTM计算结果总结}
  \label{fig:oscil}
\end{figure}

\newpage
从训练结果可以看出
\begin{itemize}
	\item drop out 对于收敛过程有影响,但基本不影响模型最终的结果
	\item 词向量个数不可以太少,增加词向量个数可以在文本转化过程中损失更少的信息,当调到256个以上后,对准确度提升不再明显
	\item 单层神经元就可以实现很好的分类效果,在本例中增加神经层数反而会导致过拟合,其中3层神经元的结果就显著弱于单层
\end{itemize}
\vspace{3mm}
因此根据奥卡姆剃刀原则,应选取准确率高的尽可能简单的模型,所以选取单层256,词向量256的LSTM模型。在测试集上精确度可以达到90.31\%.

并且我还尝试了Relu激活函数,但是结果非常差如图所示。
\begin{figure}[H]
  \centering
  \label{fig:Per6A}\includegraphics[width=0.8\textwidth]{relu_accuracy.png}\
  \caption{Re\_lu 激活函数结果}
  \label{fig:oscil}
\end{figure}
经过查找资料,虽然ReLu激活函数有很多好处,比如ReLUs的好处之一是它们避免了梯度的消失。但是LSTM一开始就被设计避开了这一点。假设没有消失梯度的问题。所以这一点并不能对lstm神经网络有提升。而从另一个方面,relu原则上是设计来训练深层的神经网络的,它的主要优势是更容易训练。而LSTM最初不是被设计成“深的”神经网络。所以反而可能会导致很多问题。


\newpage
\section{心得体会}

通过完成这次NLP作业让我对神经网络以及NLP有了更深的了解,对于处理NLP问题的一般流程有了了解,以后有时间还想继续做一些可以进一步提高准确率的工作。并且在训练过程中发现CPU占用率是满的而GPU占用率较低,估计是cpu处理batch是分开处理分部交给GPU的,如果可以并行处理应该可以大幅提高训练的速度,并也找了相关的文章,但由于时间限制没有在作业中包含这一部分。
最后感谢胡教授这学期的教学,让我收获很多,顺颂时祺。

\end{document} % DONE WITH DOCUMENT!






% IF YOU'D RATHER TYPE THE CODE, OR HAVE A SMALLER BLOCK OF CODE, USE THIS:
%\begin{lstlisting}
%if(something)
%	do this
%else
%	do this
%\end{lstlisting}

%% THIS IS FROM A DIFFERENT CLASS, BUT DEMONSTRATES MATH MODE WELL
%%%%%%%%%%%%%%%%%%%%%%%%%%%%%%
\subsection{Formulas and Overall Descriptions Used}
This part of the laboratory was done for \href{http://www.byui.edu/catalog/2004-2005/class.asp1075.htm}{Feedback Control}.  Most of this laboratory's calculations were completed and compiled by the folks at Quanser (the manufacturer of the inverted pendulum) and will give the lab a good starting place.  Below are the state equation and gain values used initially in the lab:
	\[
	\begin{bmatrix}
	\dot{\alpha} \\
	\ddot{\alpha} \\
	\dot{\theta} \\
	\ddot{\theta} \\
	\end{bmatrix}
	=
	\begin{bmatrix}
	0 & 1 & 0 & 0 \\
	81.7 & 0 & 0 & -13.9 \\
	0 & 0 & 0 & 1 \\
	39.7 & 0 & 0 & -14.4 \\
	\end{bmatrix}
	\begin{bmatrix}
	\alpha \\
	\dot{\alpha} \\
	\theta \\
	\dot{\theta} \\
	\end{bmatrix}
	+
	\begin{bmatrix}
	0 \\
	24.5 \\
	0 \\
	25.4 \\
	\end{bmatrix}
	V
	\]

	\[
	K  =
	\begin{bmatrix}
	21 & 2.8 & -2.2 & -2.0 \\
	\end{bmatrix}
	\]

Other values, such as the $\frac{\mbox{Volts}}{\mbox{Degree}}$ and $\frac{\mbox{Degrees}}{\mbox{Volt}}$ were obtained by first determining the max angle of the pendulum on both extreme sides.

Using the max angles from above, these values were determined:
	\[
	\begin{array}{l l}
		\alpha = 0.062 \frac{\mbox{Volts}}{\mbox{Degree}} \\ \\
		\alpha = 15.105 \frac{\mbox{Degrees}}{\mbox{Volt}} \\
	\end{array}
	\]

I would also like to add that in order to calibrate $\alpha$ to get a perfect vertical $= 0$, a value of $0.09$ needed to be added.  The same applies to $\theta$ where $0.322$ needs to be added.

%%%%%%%%%%%%%%%%%%%%%%%%%%%%%%
\subsection{DC Motor Transfer Function and Parameters}

Definitions:
	\begin{align*}
		\theta(t) =  Angular Position \\
		\dot{\theta}(t) =  Angular Velocity \\
		\triangle t = t_{10\%} - t_{90\%} \\
		90\% = e^{-t_{10\%}/\tau} \\
		10\% = e^{-t_{90\%}/\tau} \\
	\end{align*}

The Math:
	\begin{align*}
		\frac{s\theta(s)}{V_{a}(s)} = \frac{K}{s+P} \\
		\mbox{Let}\ V_{a}(s) = \frac{V_{0}}{s} \\  % If you'd like to have a space following any command, add "\" to the end as shown here.
		s\theta(s) = \frac{KV_{0}}{(S+P)S} = \frac{KV_{0}}{\frac{P}{S}} - \frac{\frac{KV_{0}}{P}}{s+P} \\
		L^{-1} \Rightarrow \dot{\theta}(t) = \frac{KV_{0}}{P}(1-e^{-t/(1/P)}) \\
		\dot{\theta}(t) = (\dot{\theta}_{i} - \dot{\theta}_{f})e^{-pt} + \dot{\theta}_{f} \\
	\end{align*}

Final equations:
	\begin{align}
		\label{thetadot}\dot{\theta}_{f} = \frac{KV_{0}}{P} \\
		\label{equ:tau}\frac{1}{P} = \tau = \frac{\triangle t}{ln(9)}
	\end{align}

Graphically (Refer to Equation \ref{thetadot} and Equation \ref{equ:tau}) :
	% Drawn and exported to png using Inkscape.
	\begin{figure}[h]
		\begin{center}
			\includegraphics[width=0.33\textwidth]{graph.png}
		\end{center}
	\label{graph}
	\end{figure}

% AGAIN, ANOTHER EXAMPLE FROM A DIFFERENT CLASS WHICH DEMONSTRasdATES KMAPS AND TABLES NICELY.
\newpage % I added this after viewing the completed pdf and decided to make this cosmetic change
This section consists of tables and reductions which were used in this laboratory exercise.

% This table was generated using the Calc2LaTeX macro which I mentioned earlier.
% You'll need OpenOffice installed and you'll have to download the macro online.
% If you're interested, I have a guide on how to set this up and use it on my
% blog.  http://www.derekhildreth.com/blog  Search for "LaTeX".  You'll find it.
	\begin{table}[htbp]
	\begin{center}
		\begin{tabular}{|ccc|cc|}
			\hline
			\textbf{PS} & \textbf{D} & \textbf{N} & \textbf{NS} & \textbf{P} \\ \hline
			\$0.00 & 0 & 0 & \$0.00 & 0 \\
			 & 0 & 1 & \$0.05 & 0 \\
			 & 1 & 0 & \$0.10 & 0 \\
			 & 1 & 1 & -- & -- \\ \hline
			\$0.05 & 0 & 0 & \$0.05 & 0 \\
			 & 0 & 1 & \$0.10 & 0 \\
			 & 1 & 0 & \$0.15 & 0 \\
			 & 1 & 1 & -- & -- \\ \hline
			\$0.10 & 0 & 0 & \$0.10 & 0 \\
			 & 0 & 1 & \$0.15 & 0 \\
			 & 1 & 0 & \$0.15 & 0 \\
			 & 1 & 1 & -- & -- \\ \hline
			\$0.15 & -- & -- & \$0.15 & 1 \\ \hline
			\end{tabular}
	\end{center}
	\caption{Symbolic Transition Table}
	\label{symbolic}
	\end{table}

	\begin{table}[H]
		\centering
		\subfloat[D1 = $Q_{1}$+D+$Q_{0}$N] % Caption
			{
				\karnaughmap{4}{D1:}{ {$Q_{1}$} {$Q_{0}$} {D} {N} }{001X011X111X111X}{}  % See the included kvdoc.pdf file for more details
			} \hspace{10mm} % seperate them a bit
		\subfloat[D0 = $\Bar{Q_{0}}$N + $Q_{0}\Bar{N}$ + $Q_{1}$N + $Q_{1}$D] % Caption
			{
				\karnaughmap{4}{D0:}{ {$Q_{1}$} {$Q_{0}$} {D} {N} }{010X101X011X111X}{}
			}
	  \caption{Karnaugh maps and the simplified results of the logic.}
	  \label{fig:kmaps}
	\end{table}


%%%%%%%%%%%%%%%%%%%%%%%%%%%%%%
%%%%%%%%%%%%%%%%%%%%%%%%%%%%%%
\newpage
\section{Discussion \& Conclusion}
The goal of this lab was to re-design the LED/Switch system to include a hardware timer.  By pressing eight different combinations of the three buttons, the LEDs on the board were to act in different ways using these timers. There was not a Q\&A requirement for this lab. \vspace{3mm} % I use this to seperate the paragraphs a bit.

I was able to accomplish the requirements of the lab by utilizing the \texttt{IntMgrTimerExample.c} project found within the analog devices example programs folder (and mentioned in the class lecture).  There were some stumbling blocks to overcome.  The most difficult for myself was actually getting the period of the LEDs just right.  I was able to get it very close to the 333.3\ms, 666.7\ms, and 1\s periods, but not exactly.  My first method of getting these periods right was to take the clock speed in \MHz, find the period by taking the inverse of the clock speed, and then solving for the value in hex that was needed to get the right period.  This didn't yeild very accurate results at all, and so I then went through a trial and error session until I got a value of 1.1\ms.  I used this value in hex to calculate the other periods.  The results of this method can be seen in Figure \ref{fig:oscil} above in the schematics section. \vspace{3mm}

Another observation I would like to point out is that I put all of my logic within the interrupts themselves.  I feel that this was a hacked way of doing the lab to save time and that it's probably not the best programming method.  After I was completed with my lab, I viewed other students solutions and they just seemed more elegant.  Interestingly enough, the other students weren't incredibly happy with their solution either.  If I were to go back and do this lab again, I would invest more time in both understanding how to utilze the interrupts as well as find a more elegant solution to blink the lights. \vspace{3mm}

All in all, this laboratory gave me an insight on how interrupts work and I hope to be able to apply them to following labs\ldots


\end{document} % DONE WITH DOCUMENT!


%%%%%%%%%%
PERSONAL FAVORITE LAB WRITE-UP STRUCTURE
%%%%%%%%%%
\section{Introduction}
	% No Text Here
	\subsection{Purpose}
		% Lab objective
	\subsection{Equipment}
		% Any and all equipment used (specific!)
	\subsection{Procedure}
		% Overview of the procedure taken (not-so-specific!)
\newpage
\section{Schematic Diagrams}
	% Any schematics, screenshots, block
   % diagrams used.  Possibly photos or
	% images could go here as well.
\newpage
\section{Experiment Data}
	% Depending on lab, program code would be
	% included here without the Estimated and
	% Actual Results.
	\subsection{Estimated Results}
		% Calculated. What it should be.
	\subsection{Actual Results}
		% Measured.  What it actually was.
\newpage
\section{Discussion \& Conclusion}
	% 3 Paragraphs:
		% Restate the objective of the lab
		% Discuss personal trials, errors, and difficulties
		% Conclude the lab


%%%%%%%%%%%%%%%%
COMMON COMMANDS:
%%%%%%%%%%%%%%%%
% IMAGES
begin{figure}[H]
   \begin{center}
      \includegraphics[width=0.6\textwidth]{RTL_SCHEM.png}
   \end{center}
\caption{A screenshot of the RTL Schematics produced from the Verilog code.}
\label{RTL}
\end{figure}

% SUBFIGURES IMAGES
\begin{figure}[H]
  \centering
  \subfloat[LED4 Period]{\label{fig:Per4}\includegraphics[width=0.4\textwidth]{period_led4.png}} \\
  \subfloat[LED5 Period]{\label{fig:Per5}\includegraphics[width=0.4\textwidth]{period_led5.png}}
  \subfloat[LED6 Period]{\label{fig:Per6}\includegraphics[width=0.4\textwidth]{period_led6.png}}
  \caption{Period of LED blink rate captured by osciliscope.}
  \label{fig:oscil}
\end{figure}

% INSERT SOURCE CODE
\lstset{language=Verilog, tabsize=3, backgroundcolor=\color{mygrey}, basicstyle=\small, commentstyle=\color{BrickRed}}
\lstinputlisting{MODULE.v}

% TEXT TABLE
\begin{table}
\begin{center}
\begin{tabular}{|l|c|c|l|}
	x & x & x & x \\ \hline
	x & x & x & x \\
	x & x & x & x \\ \hline
\end{tabular}
\caption{Caption}
\label{label}
\end{center}
\end{table}

% MATHMATICAL ENVIRONMENT
$ 8 = 2 \times 4 $

% CENTERED FORMULA
\[  \]

% NUMBERED EQUATION
\begin{equation}

\end{equation}

% ARRAY OF EQUATIONS (The splat supresses the numbering)
\begin{align*}

\end{align*}

% NUMBERED ARRAY OF EQUATIONS
\begin{align}

\end{align}

% ACCENTS
\dot{x} % dot
\ddot{x} % double dot
\bar{x} % bar
\tilde{x} % tilde
\vec{x} % vector
\hat{x} % hat
\acute{x} % acute
\grave{x} % grave
\breve{x} % breve
\check{x} % dot (cowboy hat)

% FONTS
\mathrm{text} % roman
\mathsf{text} % sans serif
\mathtt{text} % Typewriter
\mathbb{text} % Blackboard bold
\mathcal{text} % Caligraphy
\mathfrak{text} % Fraktur

\textbf{text} % bold
\textit{text} % italic
\textsl{text} % slanted
\textsc{text} % small caps
\texttt{text} % typewriter
\underline{text} % underline
\emph{text} % emphasized

\begin{tiny}text\end{tiny} % Tiny
\begin{scriptsize}text\end{scriptsize} % Script Size
\begin{footnotesize}text\end{footnotesize} % Footnote Size
\begin{small}text\end{small} % Small
\begin{normalsize}text\end{normalsize} % Normal Size
\begin{large}text\end{large} % Large
\begin{Large}text\end{Large} % Larger
\begin{LARGE}text\end{LARGE} % Very Large
\begin{huge}text\end{huge}   % Huge
\begin{Huge}text\end{Huge}   % Very Huge


% GENERATE TABLE OF CONTENTS AND/OR TABLE OF FIGURES
% These seem to have some issues with the "revtex4" document class.  To use, change
% the very first line of this document to "article" like this:
% \documentclass[aps,letterpaper,10pt]{article}
\tableofcontents
\listoffigures
\listoftables

% INCLUDE A HYPERLINK OR URL
\url{http://www.derekhildreth.com}
\href{http://www.derekhildreth.com}{Derek Hildreth's Website}

% FOR MORE, REFER TO THE "LINUX CHEAT SHEET.PDF" FILE INCLUDED!
